% Options for packages loaded elsewhere
\PassOptionsToPackage{unicode}{hyperref}
\PassOptionsToPackage{hyphens}{url}
\PassOptionsToPackage{dvipsnames,svgnames,x11names}{xcolor}
%
\documentclass[
  letterpaper,
  DIV=11,
  numbers=noendperiod]{scrartcl}

\usepackage{amsmath,amssymb}
\usepackage{iftex}
\ifPDFTeX
  \usepackage[T1]{fontenc}
  \usepackage[utf8]{inputenc}
  \usepackage{textcomp} % provide euro and other symbols
\else % if luatex or xetex
  \usepackage{unicode-math}
  \defaultfontfeatures{Scale=MatchLowercase}
  \defaultfontfeatures[\rmfamily]{Ligatures=TeX,Scale=1}
\fi
\usepackage{lmodern}
\ifPDFTeX\else  
    % xetex/luatex font selection
\fi
% Use upquote if available, for straight quotes in verbatim environments
\IfFileExists{upquote.sty}{\usepackage{upquote}}{}
\IfFileExists{microtype.sty}{% use microtype if available
  \usepackage[]{microtype}
  \UseMicrotypeSet[protrusion]{basicmath} % disable protrusion for tt fonts
}{}
\makeatletter
\@ifundefined{KOMAClassName}{% if non-KOMA class
  \IfFileExists{parskip.sty}{%
    \usepackage{parskip}
  }{% else
    \setlength{\parindent}{0pt}
    \setlength{\parskip}{6pt plus 2pt minus 1pt}}
}{% if KOMA class
  \KOMAoptions{parskip=half}}
\makeatother
\usepackage{xcolor}
\setlength{\emergencystretch}{3em} % prevent overfull lines
\setcounter{secnumdepth}{-\maxdimen} % remove section numbering
% Make \paragraph and \subparagraph free-standing
\ifx\paragraph\undefined\else
  \let\oldparagraph\paragraph
  \renewcommand{\paragraph}[1]{\oldparagraph{#1}\mbox{}}
\fi
\ifx\subparagraph\undefined\else
  \let\oldsubparagraph\subparagraph
  \renewcommand{\subparagraph}[1]{\oldsubparagraph{#1}\mbox{}}
\fi

\usepackage{color}
\usepackage{fancyvrb}
\newcommand{\VerbBar}{|}
\newcommand{\VERB}{\Verb[commandchars=\\\{\}]}
\DefineVerbatimEnvironment{Highlighting}{Verbatim}{commandchars=\\\{\}}
% Add ',fontsize=\small' for more characters per line
\usepackage{framed}
\definecolor{shadecolor}{RGB}{241,243,245}
\newenvironment{Shaded}{\begin{snugshade}}{\end{snugshade}}
\newcommand{\AlertTok}[1]{\textcolor[rgb]{0.68,0.00,0.00}{#1}}
\newcommand{\AnnotationTok}[1]{\textcolor[rgb]{0.37,0.37,0.37}{#1}}
\newcommand{\AttributeTok}[1]{\textcolor[rgb]{0.40,0.45,0.13}{#1}}
\newcommand{\BaseNTok}[1]{\textcolor[rgb]{0.68,0.00,0.00}{#1}}
\newcommand{\BuiltInTok}[1]{\textcolor[rgb]{0.00,0.23,0.31}{#1}}
\newcommand{\CharTok}[1]{\textcolor[rgb]{0.13,0.47,0.30}{#1}}
\newcommand{\CommentTok}[1]{\textcolor[rgb]{0.37,0.37,0.37}{#1}}
\newcommand{\CommentVarTok}[1]{\textcolor[rgb]{0.37,0.37,0.37}{\textit{#1}}}
\newcommand{\ConstantTok}[1]{\textcolor[rgb]{0.56,0.35,0.01}{#1}}
\newcommand{\ControlFlowTok}[1]{\textcolor[rgb]{0.00,0.23,0.31}{#1}}
\newcommand{\DataTypeTok}[1]{\textcolor[rgb]{0.68,0.00,0.00}{#1}}
\newcommand{\DecValTok}[1]{\textcolor[rgb]{0.68,0.00,0.00}{#1}}
\newcommand{\DocumentationTok}[1]{\textcolor[rgb]{0.37,0.37,0.37}{\textit{#1}}}
\newcommand{\ErrorTok}[1]{\textcolor[rgb]{0.68,0.00,0.00}{#1}}
\newcommand{\ExtensionTok}[1]{\textcolor[rgb]{0.00,0.23,0.31}{#1}}
\newcommand{\FloatTok}[1]{\textcolor[rgb]{0.68,0.00,0.00}{#1}}
\newcommand{\FunctionTok}[1]{\textcolor[rgb]{0.28,0.35,0.67}{#1}}
\newcommand{\ImportTok}[1]{\textcolor[rgb]{0.00,0.46,0.62}{#1}}
\newcommand{\InformationTok}[1]{\textcolor[rgb]{0.37,0.37,0.37}{#1}}
\newcommand{\KeywordTok}[1]{\textcolor[rgb]{0.00,0.23,0.31}{#1}}
\newcommand{\NormalTok}[1]{\textcolor[rgb]{0.00,0.23,0.31}{#1}}
\newcommand{\OperatorTok}[1]{\textcolor[rgb]{0.37,0.37,0.37}{#1}}
\newcommand{\OtherTok}[1]{\textcolor[rgb]{0.00,0.23,0.31}{#1}}
\newcommand{\PreprocessorTok}[1]{\textcolor[rgb]{0.68,0.00,0.00}{#1}}
\newcommand{\RegionMarkerTok}[1]{\textcolor[rgb]{0.00,0.23,0.31}{#1}}
\newcommand{\SpecialCharTok}[1]{\textcolor[rgb]{0.37,0.37,0.37}{#1}}
\newcommand{\SpecialStringTok}[1]{\textcolor[rgb]{0.13,0.47,0.30}{#1}}
\newcommand{\StringTok}[1]{\textcolor[rgb]{0.13,0.47,0.30}{#1}}
\newcommand{\VariableTok}[1]{\textcolor[rgb]{0.07,0.07,0.07}{#1}}
\newcommand{\VerbatimStringTok}[1]{\textcolor[rgb]{0.13,0.47,0.30}{#1}}
\newcommand{\WarningTok}[1]{\textcolor[rgb]{0.37,0.37,0.37}{\textit{#1}}}

\providecommand{\tightlist}{%
  \setlength{\itemsep}{0pt}\setlength{\parskip}{0pt}}\usepackage{longtable,booktabs,array}
\usepackage{calc} % for calculating minipage widths
% Correct order of tables after \paragraph or \subparagraph
\usepackage{etoolbox}
\makeatletter
\patchcmd\longtable{\par}{\if@noskipsec\mbox{}\fi\par}{}{}
\makeatother
% Allow footnotes in longtable head/foot
\IfFileExists{footnotehyper.sty}{\usepackage{footnotehyper}}{\usepackage{footnote}}
\makesavenoteenv{longtable}
\usepackage{graphicx}
\makeatletter
\def\maxwidth{\ifdim\Gin@nat@width>\linewidth\linewidth\else\Gin@nat@width\fi}
\def\maxheight{\ifdim\Gin@nat@height>\textheight\textheight\else\Gin@nat@height\fi}
\makeatother
% Scale images if necessary, so that they will not overflow the page
% margins by default, and it is still possible to overwrite the defaults
% using explicit options in \includegraphics[width, height, ...]{}
\setkeys{Gin}{width=\maxwidth,height=\maxheight,keepaspectratio}
% Set default figure placement to htbp
\makeatletter
\def\fps@figure{htbp}
\makeatother

\KOMAoption{captions}{tableheading}
\makeatletter
\@ifpackageloaded{caption}{}{\usepackage{caption}}
\AtBeginDocument{%
\ifdefined\contentsname
  \renewcommand*\contentsname{Table of contents}
\else
  \newcommand\contentsname{Table of contents}
\fi
\ifdefined\listfigurename
  \renewcommand*\listfigurename{List of Figures}
\else
  \newcommand\listfigurename{List of Figures}
\fi
\ifdefined\listtablename
  \renewcommand*\listtablename{List of Tables}
\else
  \newcommand\listtablename{List of Tables}
\fi
\ifdefined\figurename
  \renewcommand*\figurename{Figure}
\else
  \newcommand\figurename{Figure}
\fi
\ifdefined\tablename
  \renewcommand*\tablename{Table}
\else
  \newcommand\tablename{Table}
\fi
}
\@ifpackageloaded{float}{}{\usepackage{float}}
\floatstyle{ruled}
\@ifundefined{c@chapter}{\newfloat{codelisting}{h}{lop}}{\newfloat{codelisting}{h}{lop}[chapter]}
\floatname{codelisting}{Listing}
\newcommand*\listoflistings{\listof{codelisting}{List of Listings}}
\makeatother
\makeatletter
\makeatother
\makeatletter
\@ifpackageloaded{caption}{}{\usepackage{caption}}
\@ifpackageloaded{subcaption}{}{\usepackage{subcaption}}
\makeatother
\ifLuaTeX
  \usepackage{selnolig}  % disable illegal ligatures
\fi
\usepackage{bookmark}

\IfFileExists{xurl.sty}{\usepackage{xurl}}{} % add URL line breaks if available
\urlstyle{same} % disable monospaced font for URLs
\hypersetup{
  pdftitle={EDA},
  colorlinks=true,
  linkcolor={blue},
  filecolor={Maroon},
  citecolor={Blue},
  urlcolor={Blue},
  pdfcreator={LaTeX via pandoc}}

\title{EDA}
\author{}
\date{}

\begin{document}
\maketitle

\begin{Shaded}
\begin{Highlighting}[]
\FunctionTok{library}\NormalTok{(readr)}
\FunctionTok{library}\NormalTok{(tidyverse)}
\end{Highlighting}
\end{Shaded}

\begin{verbatim}
-- Attaching core tidyverse packages ------------------------ tidyverse 2.0.0 --
v dplyr     1.1.4     v purrr     1.0.2
v forcats   1.0.0     v stringr   1.5.1
v ggplot2   3.5.1     v tibble    3.2.1
v lubridate 1.9.3     v tidyr     1.3.1
-- Conflicts ------------------------------------------ tidyverse_conflicts() --
x dplyr::filter() masks stats::filter()
x dplyr::lag()    masks stats::lag()
i Use the conflicted package (<http://conflicted.r-lib.org/>) to force all conflicts to become errors
\end{verbatim}

\begin{Shaded}
\begin{Highlighting}[]
\FunctionTok{library}\NormalTok{(ggplot2)}
\FunctionTok{library}\NormalTok{(corrplot)}
\end{Highlighting}
\end{Shaded}

\begin{verbatim}
corrplot 0.94 loaded
\end{verbatim}

\subsection{Exploratory Data Analysis - Body
Fat}\label{exploratory-data-analysis---body-fat}

This (data
set){[}\url{https://dasl.datadescription.com/datafile/bodyfat}{]} lists
estimates of the percentage of body fat determined by underwater
weighing and various body circumference measurements for 252 men.

Using Siri's equation (1956), we have percent body fat, which is the
total volume of fat tissue as a proportion to the combination of lean
body tissue and fat tissue. This data set also features measurements of
age, weight, height, and the measured circumference of various sections
of the body (e.g.~neck, chest, thigh etc.)

\begin{Shaded}
\begin{Highlighting}[]
\CommentTok{\# loading data}
\NormalTok{data }\OtherTok{=} \FunctionTok{read.table}\NormalTok{(}\StringTok{"bodyfat.txt"}\NormalTok{, }\AttributeTok{header =} \ConstantTok{TRUE}\NormalTok{) }\CommentTok{\# default white space for sep value}
\FunctionTok{head}\NormalTok{(data)}
\end{Highlighting}
\end{Shaded}

\begin{verbatim}
  Density Pct.BF Age Weight Height Neck Chest Abdomen    Waist   Hip Thigh Knee
1  1.0708   12.3  23 154.25  67.75 36.2  93.1    85.2 33.54331  94.5  59.0 37.3
2  1.0853    6.1  22 173.25  72.25 38.5  93.6    83.0 32.67717  98.7  58.7 37.3
3  1.0414   25.3  22 154.00  66.25 34.0  95.8    87.9 34.60630  99.2  59.6 38.9
4  1.0751   10.4  26 184.75  72.25 37.4 101.8    86.4 34.01575 101.2  60.1 37.3
5  1.0340   28.7  24 184.25  71.25 34.4  97.3   100.0 39.37008 101.9  63.2 42.2
6  1.0502   20.9  24 210.25  74.75 39.0 104.5    94.4 37.16535 107.8  66.0 42.0
  Ankle Bicep Forearm Wrist
1  21.9  32.0    27.4  17.1
2  23.4  30.5    28.9  18.2
3  24.0  28.8    25.2  16.6
4  22.8  32.4    29.4  18.2
5  24.0  32.2    27.7  17.7
6  25.6  35.7    30.6  18.8
\end{verbatim}

\begin{Shaded}
\begin{Highlighting}[]
\FunctionTok{colnames}\NormalTok{(data)}
\end{Highlighting}
\end{Shaded}

\begin{verbatim}
 [1] "Density" "Pct.BF"  "Age"     "Weight"  "Height"  "Neck"    "Chest"  
 [8] "Abdomen" "Waist"   "Hip"     "Thigh"   "Knee"    "Ankle"   "Bicep"  
[15] "Forearm" "Wrist"  
\end{verbatim}

\subsection{Analysis}\label{analysis}

We'll look at the \textbf{body fat percentage} (BFP) as the dependent
variable that we want to predict and show its relationship against other
variables in this dataset. We'll first look at its distribution:

\begin{Shaded}
\begin{Highlighting}[]
\FunctionTok{ggplot}\NormalTok{(data, }\FunctionTok{aes}\NormalTok{(}\AttributeTok{x =}\NormalTok{ Pct.BF)) }\SpecialCharTok{+}
  \FunctionTok{geom\_histogram}\NormalTok{(}\AttributeTok{binwidth =} \FloatTok{1.5}\NormalTok{, }\AttributeTok{fill =} \StringTok{"\#da627d"}\NormalTok{, }\AttributeTok{color =} \StringTok{"white"}\NormalTok{) }\SpecialCharTok{+}
  \FunctionTok{labs}\NormalTok{(}\AttributeTok{x =} \StringTok{"Body{-}fat percentage (\%)"}\NormalTok{, }\AttributeTok{y =} \StringTok{"Frequency"}\NormalTok{) }\SpecialCharTok{+}
  \FunctionTok{theme\_minimal}\NormalTok{()}
\end{Highlighting}
\end{Shaded}

\includegraphics{Untitled_files/figure-pdf/unnamed-chunk-4-1.pdf}

\begin{Shaded}
\begin{Highlighting}[]
\FunctionTok{ggplot}\NormalTok{(data, }\FunctionTok{aes}\NormalTok{(}\AttributeTok{x =} \StringTok{""}\NormalTok{, }\AttributeTok{y =}\NormalTok{ Pct.BF)) }\SpecialCharTok{+}
  \FunctionTok{geom\_boxplot}\NormalTok{(}\AttributeTok{outlier.shape =} \ConstantTok{NA}\NormalTok{) }\SpecialCharTok{+}
  \FunctionTok{geom\_jitter}\NormalTok{(}\AttributeTok{width =} \FloatTok{0.25}\NormalTok{, }\AttributeTok{alpha =} \FloatTok{0.72}\NormalTok{, }\AttributeTok{color =} \StringTok{"\#da627d"}\NormalTok{) }\SpecialCharTok{+}
  \FunctionTok{labs}\NormalTok{(}\AttributeTok{x =} \StringTok{""}\NormalTok{, }\AttributeTok{y =} \StringTok{"BFP \%"}\NormalTok{) }\SpecialCharTok{+}
  \FunctionTok{theme\_minimal}\NormalTok{()}
\end{Highlighting}
\end{Shaded}

\includegraphics{Untitled_files/figure-pdf/unnamed-chunk-5-1.pdf}

The data looks mainly normally distributed with some out-liers in the
higher body-fat percentage space.

\subsubsection{Against other variables}\label{against-other-variables}

We can look at the relationship between BFP and things like total
weight.

\begin{Shaded}
\begin{Highlighting}[]
\NormalTok{bplot }\OtherTok{=}\NormalTok{ data }\SpecialCharTok{|\textgreater{}}
  \FunctionTok{ggplot}\NormalTok{() }\SpecialCharTok{+} \FunctionTok{aes}\NormalTok{(}\AttributeTok{x =}\NormalTok{ Weight, }\AttributeTok{y =}\NormalTok{ Pct.BF) }\SpecialCharTok{+}
  \FunctionTok{geom\_point}\NormalTok{(}\AttributeTok{color =} \StringTok{"\#ffa5ab"}\NormalTok{) }\SpecialCharTok{+}
  \FunctionTok{geom\_smooth}\NormalTok{(}\AttributeTok{method =} \StringTok{"lm"}\NormalTok{, }\AttributeTok{color =} \StringTok{"\#a53860"}\NormalTok{, }\AttributeTok{se =} \ConstantTok{FALSE}\NormalTok{) }\SpecialCharTok{+}
  \FunctionTok{labs}\NormalTok{(}\AttributeTok{x =} \StringTok{"Weight"}\NormalTok{, }\AttributeTok{y =} \StringTok{"Body Fat Percentage"}\NormalTok{) }\SpecialCharTok{+}
  \FunctionTok{theme\_minimal}\NormalTok{()}
\NormalTok{bplot}
\end{Highlighting}
\end{Shaded}

\begin{verbatim}
`geom_smooth()` using formula = 'y ~ x'
\end{verbatim}

\includegraphics{Untitled_files/figure-pdf/unnamed-chunk-6-1.pdf}

We can see there is sort of a positive correlation between weight and
body fat percentage.

\begin{Shaded}
\begin{Highlighting}[]
\NormalTok{aplot }\OtherTok{=}\NormalTok{ data }\SpecialCharTok{|\textgreater{}}
  \FunctionTok{ggplot}\NormalTok{() }\SpecialCharTok{+} \FunctionTok{aes}\NormalTok{(}\AttributeTok{x =}\NormalTok{ Height, }\AttributeTok{y =}\NormalTok{ Pct.BF) }\SpecialCharTok{+}
  \FunctionTok{geom\_point}\NormalTok{(}\AttributeTok{color =} \StringTok{"\#ffa5ab"}\NormalTok{) }\SpecialCharTok{+}
  \FunctionTok{geom\_smooth}\NormalTok{(}\AttributeTok{method =} \StringTok{"lm"}\NormalTok{, }\AttributeTok{color =} \StringTok{"\#a53860"}\NormalTok{, }\AttributeTok{se =} \ConstantTok{FALSE}\NormalTok{) }\SpecialCharTok{+}
  \FunctionTok{labs}\NormalTok{(}\AttributeTok{x =} \StringTok{"Height"}\NormalTok{, }\AttributeTok{y =} \StringTok{"Body Fat Percentage"}\NormalTok{) }\SpecialCharTok{+}
  \FunctionTok{theme\_minimal}\NormalTok{()}
\NormalTok{aplot}
\end{Highlighting}
\end{Shaded}

\begin{verbatim}
`geom_smooth()` using formula = 'y ~ x'
\end{verbatim}

\includegraphics{Untitled_files/figure-pdf/unnamed-chunk-7-1.pdf}

We can see there is not really a correlation between height and body fat
percentage.

Here we can actually see the correlation between Pct.BF and measurements
of chest, thigh and neck.

\begin{Shaded}
\begin{Highlighting}[]
\CommentTok{\# Compute the correlation matrix}
\NormalTok{cor\_matrix }\OtherTok{\textless{}{-}} \FunctionTok{cor}\NormalTok{(data[, }\FunctionTok{c}\NormalTok{(}\StringTok{"Pct.BF"}\NormalTok{, }\StringTok{"Chest"}\NormalTok{, }\StringTok{"Thigh"}\NormalTok{, }\StringTok{"Neck"}\NormalTok{)])}

\CommentTok{\# Plot the correlation heatmap}
\FunctionTok{par}\NormalTok{(}\AttributeTok{mar =} \FunctionTok{c}\NormalTok{(}\DecValTok{20}\NormalTok{, }\DecValTok{5}\NormalTok{, }\DecValTok{20}\NormalTok{, }\DecValTok{5}\NormalTok{))  }\CommentTok{\# Adjust margins (bottom, left, top, right)}
\FunctionTok{corrplot}\NormalTok{(cor\_matrix, }\AttributeTok{method =} \StringTok{"color"}\NormalTok{, }\AttributeTok{type =} \StringTok{"upper"}\NormalTok{)}
\end{Highlighting}
\end{Shaded}

\includegraphics{Untitled_files/figure-pdf/unnamed-chunk-8-1.pdf}



\end{document}
